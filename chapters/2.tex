\section{一元函数微分学}
\subsection{高阶的导数和微分}
\textbf{1188.} 设
\[y = \frac{ax + b}{cx + d}\]
求 $y^{(n)}$.

\begin{proof}
    \[y = \frac{a}{c} + \frac{bc - ad}{c} \cdot (cx + d)^{-1}\]

    设 $z = (cx + d)^{-1}$,有
    \[y^{(n)} = \frac{bc - ad}{c} \cdot z^{(n)}\]

    通过数学归纳法不难得到
    \[z^{(n)} = \frac{n!(-c)^n}{(cx + d)^{n+1}}\]

    所以
    \[y^{(n)} = (ad - bc) \cdot \frac{n! (-c)^{n-1}}{(cx + d)^{n+1}}\]
\end{proof}

\subsection{罗尔定理、拉格朗日定理和柯西定理}

\textbf{1239.} 设定义在闭区间 $[a,b]$ 上的函数 $f$ 在定义域上有连续的 $p+q$ 阶导数,在 $(a,b)$ 上有 $p + q + 1$ 阶导数,并且
\[f(a) = f'(a) = \cdots = f^{(p)}(a) = 0\]
\[f(b) = f'(b) = \cdots = f^{(q)}(b) = 0\]
求证:存在 $\xi \in (a,b)$ 使得
\[f^{(p + q + 1)}(\xi) = 0\]

\begin{proof}
    不失一般性地,认为 $p \leqslant q$.

    考虑 $f^{(n)}(x)$ 在 $E$ 上的零点个数,用 $Z(n)$ 表示.

    因为 $f(a) = f(b) = 0$,所以 $Z(0) \geqslant 2$.

    又因为 $f(a) = f(b)$,所以一定存在 $\xi \in (a,b)$ 使得 $f'(\xi) = 0$,从而 $Z(1) \geqslant 3$.

    以此类推,可以得到 $Z(p) \geqslant p + 2$.

    因为 $f^{(n)}(x)$ 的任意两个零点之间一定有 $f^{(n+1)}(x)$ 的零点,再加上当 $p \leqslant n \leqslant q$ 时 $f^{(n)}(b) = 0$,可以得到
    \[\forall n \in [p,q]: Z(n) \geqslant Z(n-1)\]
    从而
    \[Z(q) \geqslant Z(p) \geqslant p + 2\]

    然后有
    \[\forall n \in \mathbb{N}^*: Z(n) \leqslant Z(n-1) - 1\]

    所以
    \[Z(p+q+1) \geqslant Z(q) - (p + 1) \geqslant (p+2) - (p+1) = 1\]

    而这意味着在 $E$ 上有 $f^{(p+q+1)}(x)$ 的零点.
\end{proof}\vspace{9pt}

\textbf{1242.} 求证:切比雪夫-拉盖多多项式
\[L_n(x) = \mathrm{e}^x \frac{\dif^n}{\dif x^n}(x^n \mathrm{e}^{-x})\]
的所有实根都是正数.

\begin{proof}
    先求出该多项式的具体形式
    \[L_n(x) = \sum_{k=0}^{n} \biggl(\frac{n}{k}\biggr)^2 \cdot (-x)^k \cdot k!\]

    然后可以发现当 $x$ 为负数时该求和的每一项都是正数,因此它的根一定不是负数.
\end{proof}

\textbf{1262.} 求方程
\[y'(x) = \lambda y(x)\]
的通解,其中 $\lambda$ 为常数.

\begin{proof}
    若函数 $y$ 满足方程,则
    \[\bigl(y(x) \mathrm{e}^{-\lambda x}\bigr)' = y'(x) \mathrm{e}^{-\lambda x} - \lambda y(x) \mathrm{e}^{-\lambda x} = 0\]

    所以存在实数 $C$ 使得
    \[y(x) \mathrm{e}^{-\lambda x} = C \Rightarrow y(x) = C \mathrm{e}^{\lambda x}\]

    再回头检验发现对于任意的实数 $C$,函数
    \[y(x) = C \mathrm{e}^{\lambda x}\]
    都满足方程,因此该方程的通解为
    \[y(x) = C \mathrm{e}^{\lambda x}\]
    其中 $C$ 为任意常数.
\end{proof}

\subsection{增函数与减函数}
\textbf{1286.} 对于函数 $f$ 定义域 $E$ 上的一个极限点 $x_0$,若存在正实数 $\delta$ 使得在该点的去心邻域 $\mathring{U}^\delta_E(x_0)$ 上有 $x - x_0$ 与 $f(x) - f(x_0)$ 同号,则称函数 $f$ 在点 $x_0$ 是增函数.

求证:若函数 $f$ 在有限或无限的开区间 $(a,b)$ 上的每一点皆为增函数,则它在此区间上是增函数.

\begin{proof}
    运用反证法,假如 $f$ 在该区间 $(a,b)$ 上不是增函数,那么存在不同的两点 $x', x'' \in (a,b)$ 满足
    \[\bigl(x' < x''\bigr) \wedge \bigl(f(x') \geqslant f(x'')\bigr)\]

    根据题意,对于闭区间 $[x', x'']$ 上的任意一点 $x$ 都可以找到一个相应邻域 $U^\delta(x)$,这些开区间能够覆盖 $[x', x'']$,又根据开区间覆盖定理,能够找到有限个开区间
    \[U^{\delta_1}(x_1), U^{\delta_2}(x_2), \cdots, U^{\delta_n}(x_n)\]
    来覆盖 $[x', x'']$,其中 $x' \leqslant x_1 < x_2 < \cdots < x_n \leqslant x''$.

    容易验证,
    \[f(x_1) < f(x_2) < \cdots < f(x_n)\]
    \[f(x') \leqslant f(x_1)\]
    \[f(x_n) \leqslant f(x'')\]

    从而得到
    \[f(x') < f(x'')\]

    矛盾.
\end{proof}

\subsection{不定式的求值法}
\textbf{1373.1.} 求证:若函数 $f(x)$ 的二阶导数 $f''(x)$ 存在,则
\[f''(x) = \lim_{h \rightarrow 0} \frac{f(x + h) + f(x - h) - 2f(x)}{h^2}\]

\begin{proof}
    将
    \[\frac{f(x + h) + f(x - h) - 2f(x)}{h^2}\]
    看作是以 $h$ 为自变量的函数,发现当 $h \rightarrow 0$ 时分子分母都趋近于零,运用洛必达法则
    \begin{align*}
        \lim_{h \rightarrow 0} \frac{f(x + h) + f(x - h) - 2h}{h^2} &= \lim_{h \rightarrow 0} \frac{f'(x+h) - f'(x-h)}{2h}\\
        &= \lim_{h \rightarrow 0} \left[\frac{f'(x+h) - f(x)}{2h} + \frac{f'\bigl(x+(-h)\bigr) - f'(x)}{-2h}\right]\\
        &= \lim_{h \rightarrow 0} \cdot \frac{f'(x+h) - f(x)}{2h} + \lim_{h \rightarrow 0} \cdot \frac{f'(x+h) - f(x)}{2h}\\
        &= f''(x)
    \end{align*}
\end{proof}